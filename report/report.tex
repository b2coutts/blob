%%%%%%%%%%%%%%%%%%%%%%%%%%%%%%%%%%%%%%%%%
% Short Sectioned Assignment
% LaTeX Template
% Version 1.0 (5/5/12)
%
% This template has been downloaded from:
% http://www.LaTeXTemplates.com
%
% Original author:
% Frits Wenneker (http://www.howtotex.com)
%
% License:
% CC BY-NC-SA 3.0 (http://creativecommons.org/licenses/by-nc-sa/3.0/)
%
%%%%%%%%%%%%%%%%%%%%%%%%%%%%%%%%%%%%%%%%%

%----------------------------------------------------------------------------------------
%	PACKAGES AND OTHER DOCUMENT CONFIGURATIONS
%----------------------------------------------------------------------------------------

\documentclass[paper=a4, fontsize=11pt]{scrartcl} % A4 paper and 11pt font size

\usepackage[T1]{fontenc} % Use 8-bit encoding that has 256 glyphs
\usepackage{fourier} % Use the Adobe Utopia font for the document - comment this line to return to the LaTeX default
\usepackage[english]{babel} % English language/hyphenation
\usepackage{amsmath,amsfonts,amsthm} % Math packages
\usepackage{amssymb}

\usepackage[T1]{fontenc}
\usepackage[scaled]{beramono}
\usepackage{listings}

\usepackage{algpseudocode}
\usepackage{algorithm}

\usepackage{graphicx}
\DeclareGraphicsExtensions{.png}
\graphicspath{{../imgs/}{./imgs}}

\usepackage{sectsty} % Allows customizing section commands
\allsectionsfont{\normalfont\scshape} % Make all sections centered, the default font and small caps
%\renewcommand{\thesection}{}

% Shove the number into the margin, and places a dot after it.
\makeatletter
\def\@seccntformat#1{\protect\makebox[0pt][r]{\csname
the#1\endcsname.\quad}}
\makeatother

\newtheorem{theorem}{Theorem}[section]
\newtheorem{corollary}{Corollary}[theorem]
\newtheorem{lemma}[theorem]{Lemma}
% For external Lemmas and Theorems
\newtheorem*{theorem*}{Theorem}
\newtheorem*{lemma*}{Lemma}

\usepackage{fancyhdr} % Custom headers and footers
\pagestyle{fancyplain} % Makes all pages in the document conform to the custom headers and footers
\fancyhead{} % No page header - if you want one, create it in the same way as the footers below
\fancyfoot[L]{} % Empty left footer
\fancyfoot[C]{} % Empty center footer
\fancyfoot[R]{\thepage} % Page numbering for right footer
\renewcommand{\headrulewidth}{0pt} % Remove header underlines
\renewcommand{\footrulewidth}{0pt} % Remove footer underlines
\setlength{\headheight}{13.6pt} % Customize the height of the header


\numberwithin{equation}{section} % Number equations within sections (i.e. 1.1, 1.2, 2.1, 2.2 instead of 1, 2, 3, 4)
\numberwithin{figure}{section} % Number figures within sections (i.e. 1.1, 1.2, 2.1, 2.2 instead of 1, 2, 3, 4)
\numberwithin{table}{section} % Number tables within sections (i.e. 1.1, 1.2, 2.1, 2.2 instead of 1, 2, 3, 4)

\setlength\parindent{0pt} % Removes all indentation from paragraphs - comment this line for an assignment with lots of text

\newcommand{\set}[1]{\{#1\}}
\newcommand{\setoneto}[1]{\set{1,2,\ldots,#1}}

\DeclareMathOperator{\OO}{O}
\DeclareMathOperator{\fl}{float}
\DeclareMathOperator{\cut}{\delta}
% \DeclareMathOperator(\ZZ}{\ensuremath{\mathbb{Z}}}
% \DeclareMathOperator(\RR}{\ensuremath{\mathbb{R}}}
%\DeclareMathOperator{\lg}{lg}
\newcommand{\ceil}[1]{\left\lceil#1\right\rceil}
\newcommand{\abs}[1]{\left\lvert#1\right\rvert}
\newcommand{\floor}[1]{\left\lfloor#1\right\rfloor}
\newcommand{\trans}[1]{#1^\intercal}
\newcommand{\opt}[1]{\mathbf{#1}}
%----------------------------------------------------------------------------------------
%	TITLE SECTION
%----------------------------------------------------------------------------------------

\newcommand{\horrule}[1]{\rule{\linewidth}{#1}} % Create horizontal rule command with 1 argument of height

\title{\
    \normalfont\normalsize
    \textsc{University of Waterloo} \\ [25pt] % Your university, school and/or department name(s)
    \horrule{0.5pt} \\[0.4cm] % Thin top horizontal rule
    \huge CS 759 Report:\\
    Killing Time \\
    \horrule{2pt} \\[0.5cm] % Thick bottom horizontal rule
}

\author{Theo Belaire \\ 20415730 \\ Bryan Coutts \\ 20428420} % Your name

\date{\normalsize\today} % Today's date or a custom date

\begin{document}

\maketitle % Print the title


Some terminology, I use the term \textit{included} point to refer to points
that are to be included inside the blob, and \textit{excluded} points to refer
to points that must be on the exterior of the blob.

Unless specified, polygon can mean non-convex polygon.


\section{Generation of the Convex Hull}
Our code uses the giftwrap algorithm to generate the convex hull.
This runs in $\O(i^2)$ time, where $i$ is the number of points included in the
set.

It works by picking the leftmost point, and calculating the angle that is
formed with each other point in the set, and picking the point that forms
the angle closest to a straight line with the previous point.

% TODO nice diagram of in progress giftwrap?
At the conclusion of this step, we have a list of included points in clockwise
order, which forms our polygon.  Currently it's convex.

\section{Chunkification}
To test if a point is within any polygon, we can use the Jordon Curve Theorem.
This states that if we trace a ray from the point in any direction, it
will cross an edge of the polygon an even number of times if and only if
the point is outside the polygon.

This is computationally simple to check, and we use it as a subroutine.

For each excluded point $p$, we check if it's in the polygon, and if it is,
we attempt to find the edge that it is closest to.
We can calculate the normal to an edge $uv$ easily enough, since there are
only two possible unit vectors orthogonal to the vector $\vec{uv}$.  Since
we know our points are ordered by clockwise order, there is exactly one outwards
pointing unit vector, call it $\vec{n}$.

To check the distance between a point $p$ and the edge 
$uv$, we can consider the value
\[ \vec{pu} \cdot \vec{n} \]
which is just the distance projected onto the normal.

We also need to check that the point $p$ is closest to a point on the line
segment, not just the infinite line, so we can compare
\[ \vec{pu} \cdot \vec{vu} \]
and
\[ \vec{pv} \cdot \vec{uv} \]
which should both be positive if $p$ is between the two points.

% TODO picture.

Once we've found the unique closest line segment to an exterior point,
we insert it into the list between the two.

\section{Absorbing points near lines}

\section{Calculating Radii}

\section{Uncrossing}

\section{Arc computation}
Once we have a final boundary, and radii, we can draw the blob.

For each edge $uv$, we consider a pair of circles centered at
$u$ and $v$, with the radii $r_u$ and $r_v$.

Consider as if we were to take a piece of string and wrap it around
the outside of these circles.
It would have three segments,
first when it was in contact with the first circle,
then it would leave and travel along a line tangent to both circles,
then it would arrive at the second circle
and travel along it before leaving somewhere on the other side.




\begin{figure}[h]
\includegraphics[width=\textwidth]{7/folder}
\caption{''It's number 7''}
\end{figure}
\end{document}
